\makeatletter
\newcommand\input@path{{smile/}}
\makeatother

\documentclass[coloraccent=darkmaroon]{poster}

\title{\gradient{Building a data flow graph for Java}}
\author{Lukas Pietzschmann, Denis Neumüller, Florian Sihler}
\uni{Ulm University}
\institute{Institute of Software Engineering and Programming Languages}
\date{07/28/2023}
\logoleft{logos/splogo.png}
\logoright{logos/uulmlogo.png}

\usepackage{emoji}
\usepackage{gradient}
\usepackage{textcomp}

\makeatletter
\newcommand\currentfontsize{\fontsize{\f@size}{\f@baselineskip}\selectfont}
\makeatother
\newminted{java}{fontsize=\currentfontsize, numbers=none, frame=none}
\newmintinline{java}{fontsize=\currentfontsize}
\newmintinline[bigjava]{java}{fontsize=\large, escapeinside=||}
\usemintedstyle{stata}

\columnseprule0pt
\newcommand\gradient[1]{\gradientRGB{#1}{133,29,46}{102,51,77}}
\newcommand*\circled[1]{\tikz[baseline={([yshift=-1.5pt]char.base)}]{\node[pill, opacity=0.8, inner sep=3pt](char){\small#1};}}
\let\oldemph\emph
\renewcommand\emph[1]{\oldemph{\gradient{#1}}}

\tikzstyle{coolgraphnode} = [roundednode, fill=lightgray, draw=none, inner sep=2mm]

\addtolength{\skip\footins}{-1ex}

\usepackage{pgfpages}

\begin{document}
\pgfpagesuselayout{resize to}[a1paper, landscape]
\maketitle
\begin{multicols}{3}
	\begin{mybox}{accent}{}{enhanced, watermark text={\fontspec{Symbola}\symbol{"2615}}, watermark opacity=0.3, height=5.5cm}
		\begin{javacode*}{escapeinside=||, fontsize=\Large}
			void foo(int[] arr|\circled{1}|){
				int len|\circled{2}| = arr|\circled{3}|.length;
				for(int i = 0; i < len; ++i) {
					System.out.println(arr[i]);
				}
			}
		\end{javacode*}
	\end{mybox}%
	\par\columnbreak%
	\begin{mybox}{red!75!black}{\centering\Large\bfseries The Problem}{enhanced, watermark tikz={\draw[line width=2mm] circle (1cm)
		node{\fontsize{20mm}{20mm}\selectfont?};}, watermark opacity=0.3, height=5.5cm, split=0.382, valign=center, valign lower=center}
		\centering\Large Is \javainline{len} dependent on \javainline{arr}?
		\tcblower
		\Large Yes, because there's a path between \javainline{int len} \tikz[baseline={([yshift=-0.5pt]char.base)}]{\node[pill, opacity=0.8, inner sep=3pt](char){2};} and
		\javainline{int[] arr} \tikz[baseline={([yshift=-0.5pt]char.base)}]{\node[pill, opacity=0.8, inner sep=3pt](char){1};}.
	\end{mybox}
	\par\columnbreak
	\begin{mybox}{accent}{}{enhanced, watermark tikz={\draw[line width=2mm] circle (1cm)
		node{\fontsize{20mm}{20mm}\selectfont !};}, watermark opacity=0.3, height=5.5cm}
		\centering
		\begin{tikzpicture}
			\node [coolgraphnode] (fa) [] {\bigjava{int[] arr}};
			\node [coolgraphnode] (ar) [below=of fa] {\bigjava{arr.length}};
			\node [coolgraphnode] (fl) [left=of ar, xshift=-1.5cm] {\bigjava{int len}};
			\begin{scope}[opacity=0.3]
				\node [coolgraphnode] (lr) [below=of fl] {\bigjava{len}};
				\node [coolgraphnode] (lol) [below=of ar] {\bigjava{arr[i]}};
			\end{scope}

			\node [pill] at (fa.north east) {\small1};
			\node [pill] at (fl.north east) {\small2};
			\node [pill] at (ar.north east) {\small3};

			\draw [arrow] (ar) to node [anchor=west, color=darkgray] {read from} (fa);
			\draw [arrow, opacity=0.3] (lr) to node [anchor=west, color=darkgray, opacity=0.3] {read from} (fl);
			\draw [arrow] (fl) to node [midway, above, sloped, color=darkgray] {defined by} (ar);
			\draw [arrow, opacity=0.3] (lol) to [out=5, in=-5] node [midway, below,
			sloped, color=darkgray, opacity=0.3, rotate=6] {read from} (fa);
		\end{tikzpicture}
	\end{mybox}%
\end{multicols}

\begin{multicols}{4}
	\section*{The Algorithm}
	We keep a set of declarations whose data flow we are interested in --- the \emph{active
	set}. When traversing the AST, we look at every piece of code, but we're especially
	interested in
	\begin{enumerate*}
		\item variables and
		\item assignments,
	\end{enumerate*}
	as they primarily influence the data flow.
	\par
	For every variable we encounter, we get its declaration and add it to the active
	set\footnote{If an element is already active, the set will kindly reject it.}.
	Additionally, we create a new node inside the graph. This node can have one of two
	types:
	\begin{enumerate*}
		\item \emph{Def}, if the element is a declaration or on the left-hand side of an
			assignment and
		\item \emph{Use}, otherwise.
	\end{enumerate*}
	\par
	Now we can hook up the newly added node. For this purpose, we get all recently added
	elements that \emph{share a declaration} with the current element ---
	\oldemph{e.g.}, in the figure on the right, all \javainline{a}'s share a
	declaration. We then connect the newly added node to all recently added nodes, only
	\emph{omitting Use -> Use} edges.
	\par
	Assignments are already largely handled by the rules above. Variables on the left
	and right-hand side are correctly inserted, we just have to connect the assignee to
	active elements in the assignment. This is done by getting all active elements from
	the right-hand side and drawing an edge from the \emph{left-hand side to them}.
	\par
	The types of two nodes alone lets us infer the type of the edge connecting them. We
	distinguish between three edge types:
	\begin{enumerate*}
		\item \emph{Read From}: \javainline{Use}~\textrightarrow~\javainline{Def},
		\item \emph{Defined By}: \javainline{Def}~\textrightarrow~\javainline{Use}, and
		\item \emph{Occurrence}: \javainline{Def}~\textrightarrow~\javainline{Def}.
	\end{enumerate*}
	\par
	Getting the data flow to, from, or between elements can be narrowed down to a
	\emph{graph reachability problem}.
	\begin{enumerate*}
		\item \emph{To}: We return all elements that can be reached from the given element.
		\item \emph{From}: We return all elements that can reach the given element.
		\item \emph{Between}: We return all paths between two elements.
	\end{enumerate*}
	\vspace*\fill
	\columnbreak

	\begin{minipage}{\dimexpr2\columnwidth+\columnsep}
		\section*{An Example}
		\begin{mybox}{accent}{}{sidebyside, sidebyside align=center}
			\begin{javacode*}{fontsize=\large}
				public int foo(int n, int i) {
					int a = n + 5;
					if(Math.random() > 0.5) {
						a = i;
					} else {
						a = a + 2;
					}
					return a;
				}
			\end{javacode*}
			\tcblower
			\begin{tikzpicture}
				\node [coolgraphnode] (at) [] {\bigjava{|\textbf{a}| = i}};
				\node [coolgraphnode] (ao) [right=of at, xshift=1.3cm]
				{\bigjava{int |\textbf{a}| = n + 5}};

				\node [coolgraphnode] (n) [above=of ao] {\bigjava{int |\phantom{i}\llap{a}| = |\textbf{n}| + 5}};
				\node [coolgraphnode] (i) [above=of at] {\bigjava{a = |\textbf{i}|}};

				\node [coolgraphnode] (intn) [above=of n] {\bigjava{int n}};
				\node [coolgraphnode] (inti) [above=of i] {\bigjava{int i}};

				\node [coolgraphnode] (ar) [below=of ao] {\bigjava{a = |\textbf{a}| + 2}};

				\node [coolgraphnode] (ae) [below=of ar] {\bigjava{|\textbf{a}| = a + 2}};

				\node [coolgraphnode] (ra) [below=of ae] {\bigjava{return |\textbf{a}|}};

				\draw [arrow] (ra) to node [anchor=west, color=darkgray] {read from} (ae);
				\draw [arrow] (ra) to [out=180, in=-90] node [near end, sloped, below, color=darkgray] {read from} (at);
				\draw [arrow] (at) to node [anchor=west, color=darkgray] {defined by} (i);
				\draw [arrow] (i) to node [anchor=west, color=darkgray] {read from} (inti);
				\draw [arrow] (at) to node [anchor=south, color=darkgray] {occurrence} (ao);
				\draw [arrow] (ae) to node [anchor=east, color=darkgray] {defined by} (ar);
				\draw [arrow] (ae.north east) to [in=-45, out=45] node [pos=0.6,
				rotate=14, below, color=darkgray, rotate=76] {occurrence} (ao.south east);
				\draw [arrow] (ar) to node [anchor=east, color=darkgray] {read from} (ao);
				\draw [arrow] (ao) to node [anchor=west, color=darkgray] {defined by} (n);
				\draw [arrow] (n) to node [anchor=west, color=darkgray] {read from} (intn);
			\end{tikzpicture}
		\end{mybox}

		\section*{Future Work}
		While basic cases are already handled well, a variety of situations are taken
		care of poorly or not at all. Here are some areas we want to improve in:
		\begin{description}
			\item [Methods] When a method is called, we definitely want to trace the
				data flow of all arguments through the method. We currently have basic
				support for this, but work still needs to be put into it.
			\item [Recursion] If a method calls itself --- either directly or indirectly
				--- we need to recognize this when generating and traversing the data
				flow graph.
			\item [Arrays] We currently treat arrays as a single \oldemph{thing}.
				Ideally, we would keep track of every index that is accessed.
			\item [Control Structures] We currently only treat \javainline{if-else}
				statements correctly. Support for other control structures
				\hbox{(\oldemph{e.g.}, loops)} is definitely needed.
		\end{description}
	\end{minipage}
	\par\vspace*\fill
	\columnbreak
	\null
	\columnbreak

	\section*{Project Organization}
	\paragraph{Organization}
	We're doing \emph{weekly meetings} to review the current progress, discuss decisions
	and / or potential issues, and prioritize and schedule future work. If important
	decisions were made in our meeting, we record them in a \emph{wiki} so that we can
	still understand later why we decided the way we did.

	\paragraph{Technology Stack}
	The project is developed using \hbox{\emph{Java 17}}. Additionally, we employ \emph{Gradle}
	(Groovy) as a build and dependency management system.
	\par
	The current AlDeSCo-Prototype uses \emph{Spoon} to parse and analyze any given Java
	source code. We then traverse Spoon's AST in order to build the data flow graph,
	which is represented by \emph{JGraphT}'s data structures. The actual spanning of the
	data flow graph is completely done by ourselves without the help of a library.

	\paragraph{QA}
	In order to assure that no functionality breaks over time, we use a
	small\footnote{Only the test suite targeting the data flow is currently kinda small.
	AlDeSCo's main test suite contains $\approx$ 470 tests.} but (hopefully
	\emoji{wink}) expanding \emph{test suite}.
	\par
	To ensure the employment of best practices, we often \emph{review} newly added code
	in our weekly meetings. This massively helps in keeping the code easy to understand
	and well readable. However, if we overlook a small error, it will most likely be
	caught by TeamScale, our \emph{static code analysis} platform.
	\par
	We also use \emph{assertions} whenever possible to make sure our mental model aligns
	with the actual execution of the code.

	\paragraph{Documentation}
	The above-mentioned GitLab \emph{wikis} are not only used for meeting protocols, but
	also to document important aspects of the code. However, most of our documentation
	is written inline using the \emph{JavaDoc} syntax.
\end{multicols}
\end{document}